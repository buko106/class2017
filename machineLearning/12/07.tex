\documentclass[a4j]{jarticle}
%%  packages
\usepackage{amsmath,amssymb,ascmac}
\usepackage{bm}
\usepackage[dvipdfmx]{graphicx}
\usepackage{listings}
\usepackage[english]{babel}
\lstset{
 	%枠外に行った時の自動改行
 	breaklines = true,
 	%標準の書体
        basicstyle=\ttfamily\footnotesize,
        commentstyle=\footnotesize\bfseries,
        keywordstyle=\footnotesize\bfseries,
 	%枠 "t"は上に線を記載, "T"は上に二重線を記載
	%他オプション:leftline,topline,bottomline,lines,single,shadowbox
 	frame = single,
 	%frameまでの間隔(行番号とプログラムの間)
 	framesep = 5pt,
 	%行番号の位置
 	numbers = left,
	%行番号の間隔
 	stepnumber = 1,
	%タブの大きさ
 	tabsize = 4,
 	%キャプションの場所("tb"ならば上下両方に記載)
 	captionpos = t
}

%% math commands
\let \ds \displaystyle
\newcommand{\idiff}[3]{
  \frac{d^{#1} #2}{d #3^{#1}}
}
\newcommand{\diff}[3]{
  \frac{\mathrm{d}^{#1} #2}{\mathrm{d} #3^{#1}}
}
\newcommand{\pdiff}[3]{
  \frac{\partial^{#1} #2}{\partial #3^{#1}}
}



%% title configuration
\title{統計的機械学習 ID:07}
\author{05-161026 平出一郎}
\date{\today}


%% headings
\pagestyle{headings}
\markright{統計的機械学習 ID:07}




\begin{document}
%%  begin title page
\thispagestyle{empty}
\maketitle
\pagebreak

スライドの導出より

\begin{align*}
q(u_j) & \varpropto \exp \left( -\frac{1}{2\sigma^2}\left( -2u_j^\top \sum_{i:(j,i)\in O}r_{j,i}E_{q(v_i)}[v_i] + u_j^\top\left( \sum_{i:(j,i)\in O} E_{q(v_i)}[vv^\top] + \sigma^2 \Lambda_u \right) u_j \right) \right) \\
 \intertext{$\mu_{u,j},V_{u,j}$を用いて整理すると、}
 & = \exp \left( -\frac{1}{2\sigma^2}\left( -2u_j^\top \sigma^2 V_{u,j}^{-1} \mu_{u,j} + u_j^\top \sigma^2 V_{u,j}^{-1} u_j \right) \right) \\
 & \varpropto \exp \left( -\frac{1}{2} (u_j - \mu_{u,j})^\top V_{u,j}^{-1} (u_j - \mu_{u,j}) \right) \\
\intertext{比例定数を計算するまでもなく、以下が成り立つことがわかる。}
 q(u_j) &= \mathcal{N}\left( u_j | \mu_{u,j},V_{u,j}\right)
\end{align*}
$u_j$と$v_i$を入れ替えて同様に計算すれば、 $ q(v_i) = \mathcal{N}\left( v_i | \mu_{v,i},V_{v,i}\right) $ も導かれる。


\end{document}


