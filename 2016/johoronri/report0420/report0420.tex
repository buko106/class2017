\documentclass{article}
\usepackage{amsmath}
\begin{document}

\section{束の定義を述べよ}

半順序集合$(X,\leq)$であって、以下の条件を満足するもの
\begin{itemize}
  \item 最大元$\top_X$が存在する。 
  \item 最小元$\bot_X$が存在する。
  \item 任意の$x,y \in X$に対して、$x \wedge y,x \vee y$が存在する。
\end{itemize}

\section{$X=\{0,1,2\}$のとき$\mathcal{P}(\mathcal{P}(X))$の濃度を示せ}

$|X|=3$なので$|\mathcal{P}(X)|=2^3=8$。
したがって、$|\mathcal{P}(\mathcal{P}(X))|=2^8=256$

\section{$X \neq \emptyset$ とする。 $\emptyset \rightarrow X,X \ \rightarrow \emptyset$は存在するか}

二項関係を考えると$\emptyset \times X = \emptyset,X \times \emptyset = \emptyset$であり、前者では$\emptyset$には元が存在しないゆえ任意の元の行き先が定まっているので、関数$\emptyset \rightarrow X$は存在し、また$|\mathcal{P}(\emptyset \times X)|=1$なので一意である。後者では$X$の元の行きさきが定まっていないので関数$X \rightarrow \emptyset$は存在しない。

\section{$\mathcal{P}(X \times Y ) \cong (\mathcal{P}(Y))^X$を示せ}

$f:\mathcal{P}(X \times Y ) \rightarrow (\mathcal{P}(Y))^X$を
$f(S)=((X \ni) x \mapsto \{y|(x,y)\in S \}) $、
$g:(\mathcal{P}(Y))^X \rightarrow \mathcal{P}(X \times Y )$を
$g(F)=\{(x,y)|y \in F(x)\}$
でさだめると、

任意の$S \in \mathcal{P}(X \times Y)$に対し$(S\subseteq X \times Y)$、$(g \circ f)(S)=g(f(S))=\{(x,y)|y \in (f(S))(x)\}=\{(x,y)|y \in \{y'|(x,y')\in S\}\}=\{(x,y)|(x,y)\in S\}=S$であるので、$g \circ f = \rm id_{\mathcal{P}(X \times Y)}$ 

任意の$F \in (\mathcal{P}(Y))^X,x\in X$に対して$((f \circ g)(F))(x)=(f(g(F)))(x)=(f(\{(x,y)|y \in F(x)\}))(x)=\{y|(x,y)\in\{(x,y)|y \in F(x)\}\}=\{y|y\in F(x)\}=F(x)$であるので、$f \circ g = \rm id_{(\mathcal{P}(Y))^X}$

以上より$g=f^{-1}$で同型が言える。

\end{document}
