\documentclass[a4j]{jarticle}
%%  packages
\usepackage{amsmath,amssymb,ascmac}
\usepackage{bm}
\usepackage[dvipdfmx]{graphicx}
\usepackage{listings}
\usepackage[english]{babel}
\lstset{
 	%枠外に行った時の自動改行
 	breaklines = true,
 	%標準の書体
        basicstyle=\ttfamily\footnotesize,
        commentstyle=\footnotesize\bfseries,
        keywordstyle=\footnotesize\bfseries,
 	%枠 "t"は上に線を記載, "T"は上に二重線を記載
	%他オプション:leftline,topline,bottomline,lines,single,shadowbox
 	frame = single,
 	%frameまでの間隔(行番号とプログラムの間)
 	framesep = 5pt,
 	%行番号の位置
 	numbers = left,
	%行番号の間隔
 	stepnumber = 1,
	%タブの大きさ
 	tabsize = 4,
 	%キャプションの場所("tb"ならば上下両方に記載)
 	captionpos = t
}

%% math commands
\let \ds \displaystyle
\newcommand{\idiff}[3]{
  \frac{d^{#1} #2}{d #3^{#1}}
}
\newcommand{\diff}[3]{
  \frac{\mathrm{d}^{#1} #2}{\mathrm{d} #3^{#1}}
}
\newcommand{\pdiff}[3]{
  \frac{\partial^{#1} #2}{\partial #3^{#1}}
}



%% title configuration
\title{統計的機械学習 ID:05}
\author{05-161026 平出一郎}
\date{\today}


%% headings
\pagestyle{headings}
\markright{統計的機械学習 ID:05}




\begin{document}
%%  begin title page
\thispagestyle{empty}
\maketitle
\pagebreak

\begin{align*}
\int q_{t-1}(z_2)\log p( z_1,z_2 | \bm{\mu},\Sigma) dz_2 &= const - \frac{1}{2}\int q_{t-1}(z_2)\left\{ \Lambda_{11}(z_1-\mu_1)^2 \right. \\
&\quad\quad\quad\quad\quad\quad\quad\quad \left. + 2\Lambda_{12}(z_1-\mu_1)(z_2-\mu_2) +\Lambda_{22}(z_2-\mu_2) \right\} dz_2 \\
&=const' - \frac{1}{2} \left\{ \Lambda_{11}\left(\int q_{t-1}(z_2)dz_2\right)(z_1-\mu_1)^2\right.\\
&\quad\quad\quad\quad\quad\quad\quad\quad \left.+2\Lambda_{12}\left(\int q_{t-1}(z_2)z_2dz_2 - \mu_2 \right)(z_1-\mu_1) \right\} \\
&=const'' - \frac{1}{2}\left\{ z_1-\mu_1+\Lambda_{11}^{-1}\Lambda_{12} \left( \int q_{t-1}(z_2)z_2dz_2 - \mu_2  \right) \right\}^2
\end{align*}


よって$q_t(z_1) = \mathcal{N} ( z_1 | \mu_1- \Lambda_{11}^{-1}\Lambda_{12} ( \int q_{t-1}(z_2)z_2dz_2 - \mu_2) ,\Lambda_{11}^{-1} ) $となる。


同様に$q_t(z_2) = \mathcal{N} ( z_2 | \mu_2- \Lambda_{22}^{-1}\Lambda_{12} ( \int q_{t}(z_1)z_1dz_1 - \mu_1 ) ,\Lambda_{22}^{-1}  )$も成り立つ。

$q_t(z_1) = \mathcal{N}(z_1 |  m^{(t)}_1 \Lambda_{11}^{-1} )$,$q_t(z_2) = \mathcal{N}(z_2 |  m^{(t)}_2 \Lambda_{22}^{-1} )$と書けば,更新式は次のようになる。

$m^{(t)}_1 \leftarrow \mu_1-\Lambda_{11}^{-1}\Lambda_{12} ( m_2^{(t-1)} - \mu_2)$


$m^{(t)}_2 \leftarrow \mu_2-\Lambda_{22}^{-1}\Lambda_{12} ( m_1^{(t)} - \mu_1)$



\end{document}


