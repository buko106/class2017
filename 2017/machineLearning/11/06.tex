\documentclass[a4j]{jarticle}
%%  packages
\usepackage{amsmath,amssymb,ascmac}
\usepackage{bm}
\usepackage[dvipdfmx]{graphicx}
\usepackage{listings}
\usepackage[english]{babel}
\lstset{
 	%枠外に行った時の自動改行
 	breaklines = true,
 	%標準の書体
        basicstyle=\ttfamily\footnotesize,
        commentstyle=\footnotesize\bfseries,
        keywordstyle=\footnotesize\bfseries,
 	%枠 "t"は上に線を記載, "T"は上に二重線を記載
	%他オプション:leftline,topline,bottomline,lines,single,shadowbox
 	frame = single,
 	%frameまでの間隔(行番号とプログラムの間)
 	framesep = 5pt,
 	%行番号の位置
 	numbers = left,
	%行番号の間隔
 	stepnumber = 1,
	%タブの大きさ
 	tabsize = 4,
 	%キャプションの場所("tb"ならば上下両方に記載)
 	captionpos = t
}

%% math commands
\let \ds \displaystyle
\newcommand{\idiff}[3]{
  \frac{d^{#1} #2}{d #3^{#1}}
}
\newcommand{\diff}[3]{
  \frac{\mathrm{d}^{#1} #2}{\mathrm{d} #3^{#1}}
}
\newcommand{\pdiff}[3]{
  \frac{\partial^{#1} #2}{\partial #3^{#1}}
}



%% title configuration
\title{統計的機械学習 ID:06}
\author{05-161026 平出一郎}
\date{\today}


%% headings
\pagestyle{headings}
\markright{統計的機械学習 ID:06}




\begin{document}
%%  begin title page
\thispagestyle{empty}
\maketitle
\pagebreak


代表して$q(u_1)$の関数だと思って整理すると
\begin{align*}
\log p(\{x_{i,j}\}) &= \log\int p(\{x_{i,j}\},u_{1:4},v_{1:4})du_{1:4}dv_{1:4} \\
&\geq \int q(u_{1:4})q(v_{1:4})\log \frac{p(\{x_{i,j}\},u_{1:4},v_{1:4})}{q(u_{1:4})q(v_{1:4})}du_{1:4}dv_{1:4} \\
&= \int q(u_1) \left( \int \prod_{i=2}^4q(u_i)\prod_{j=1}^4q(v_j) \log p(\{x_{i,j}\},u_{1:4},v_{1:4}) du_{2:4}dv_{1:4} \right) du_1 \\
&\quad\quad-\int q(u_1) \log q(u_1)du_1 + const
\end{align*}


となる。

変分法を用いて積分の中身の$q(u_1)$での微分$=0$を考えることによって、


$$ q(u_1) \propto \exp\left( \int \prod_{i=2}^4q(u_i)\prod_{j=1}^4q(v_j) \log p(\{x_{i,j}\},u_{1:4},v_{1:4}) du_{2:4}dv_{1:4} \right) $$

が成り立つ。

\begin{align*}
&\int \prod_{i=2}^4q(u_i)\prod_{j=1}^4q(v_j) \log p(\{x_{i,j}\},u_{1:4},v_{1:4}) du_{2:4}dv_{1:4} \\
=& - \frac{1}{2} \int \prod_{i=2}^4q(u_i)\prod_{j=1}^4q(v_j) \left( \sum_{\exists(i,j)}(x_{i,j}-u_iv_j)^2 + \sum_{i=1}^4 u_i^2 + \sum_{j=1}^4 v_j^2 \right)du_{2:4}dv_{1:4}
\end{align*}

カッコの中が$u_1$の係数正の二次式になる。よって式の形から$q(u_1) = \mathcal{N}(u_1 | \mu_{u_1}, \sigma^2_{u_1})$となることがわかる。

同様にして、$q(u_i) = \mathcal{N}(u_i | \mu_{u_i}, \sigma^2_{u_i})$,$q(v_i) = \mathcal{N}(v_i | \mu_{v_i}, \sigma^2_{v_i})$と書けることを利用して変形すると、


\begin{align*}
=& - \frac{1}{2} \left\{ \left(\int \prod_{j=1}^4q(v_j) \left( \sum_{\exists(1,j)}v_j^2 + 1 \right)dv_{1:4} \right) u_1^2  -2 \left( \int \prod_{j=1}^4q(v_j) \sum_{\exists(1,j)} v_jx_{1,j} dv_{1:4} \right) u_1 + const \right\} \\
=& - \frac{1}{2} \left\{ \left( \sum_{\exists(1,j)} \left(\sigma^2_{v_j}+\mu^2_{v_j} \right) + 1 \right) u_1^2 -2 \left( \sum_{\exists(1,j)} \mu_{v_j}x_{1,j} \right)u_1 + const \right\}
\end{align*}

よって更新式は


$\ds \sigma^2_{u_1} \leftarrow \left( \sum_{\exists(1,j)} \left(\sigma^2_{v_j}+\mu^2_{v_j} \right) + 1 \right)^{-1}$


$\ds \mu_{u_1} \leftarrow \left( \sum_{\exists(1,j)} \left(\sigma^2_{v_j}+\mu^2_{v_j} \right) + 1 \right)^{-1}\left( \sum_{\exists(1,j)} \mu_{v_j}x_{1,j} \right)$


$u_i,v_j$に対しても同様である。

pythonを用いて、ランダムな初期値から収束するまで更新を繰り返した。

\newpage 
\lstinputlisting[caption=result]{result06.txt}
\lstinputlisting[caption=source code]{06.py}

\end{document}


