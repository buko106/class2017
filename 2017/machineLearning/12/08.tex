\documentclass[a4j]{jarticle}
%%  packages
\usepackage{amsmath,amssymb,ascmac}
\usepackage{bm}
\usepackage[dvipdfmx]{graphicx}
\usepackage{listings}
\usepackage[english]{babel}
\lstset{
 	%枠外に行った時の自動改行
 	breaklines = true,
 	%標準の書体
        basicstyle=\ttfamily\footnotesize,
        commentstyle=\footnotesize\bfseries,
        keywordstyle=\footnotesize\bfseries,
 	%枠 "t"は上に線を記載, "T"は上に二重線を記載
	%他オプション:leftline,topline,bottomline,lines,single,shadowbox
 	frame = single,
 	%frameまでの間隔(行番号とプログラムの間)
 	framesep = 5pt,
 	%行番号の位置
 	numbers = left,
	%行番号の間隔
 	stepnumber = 1,
	%タブの大きさ
 	tabsize = 4,
 	%キャプションの場所("tb"ならば上下両方に記載)
 	captionpos = t
}

%% math commands
\let \ds \displaystyle
\newcommand{\idiff}[3]{
  \frac{d^{#1} #2}{d #3^{#1}}
}
\newcommand{\diff}[3]{
  \frac{\mathrm{d}^{#1} #2}{\mathrm{d} #3^{#1}}
}
\newcommand{\pdiff}[3]{
  \frac{\partial^{#1} #2}{\partial #3^{#1}}
}



%% title configuration
\title{統計的機械学習 ID:08}
\author{05-161026 平出一郎}
\date{\today}


%% headings
\pagestyle{headings}
\markright{統計的機械学習 ID:08}




\begin{document}
%%  begin title page
\thispagestyle{empty}
\maketitle
\pagebreak



\begin{align*}
\intertext{$\Lambda_u$に着目すると、}
L[q(u_{1:J},v_{i:j});\Lambda_u,\Lambda_v,\sigma^2] &= \int q(u_{1:J}) \log p(u_{1:J}|\Lambda_u) du_{1:J} &+ const\\
&= \int \prod_{j=1}^Jq(u_j) \sum_{j=1}^J \log p(u_j|\Lambda_u) du_{1:J} &+ const \\
&= \sum_{j=1}^J \int \mathcal{N}(u_j|\mu_{u,j},V_{u,j}) \log \mathcal{N}(u_j|0,\Lambda_u)du_j &+const \\
&= \sum_{j=1}^J \int \mathcal{N}(u_j|\mu_{u,j},V_{u,j}) \left( -\frac{1}{2} \log |\Lambda_u| - \frac{1}{2} u_j^\top \Lambda_u^{-1}u_j \right) du_j &+ const \\
&= -\frac{J}{2} \log |\Lambda_u| -\frac{1}{2}\sum_{j=1}^J\int\mathcal{N}(u_j|\mu_{u,j},V_{u,j})\sum_{k=1}^K\frac{u_{j,k}^2}{\rho_{u,k}^2}du_j &+ const \\
\intertext{ガウス分布の周辺化により、}
&= -J \sum_{k=1}^K \log \rho_{u,k} -\frac{1}{2}\sum_{j=1}^J\sum_{k=1}^K\rho_{u,k}^{-2}\int\mathcal{N}(u_{j,k}|\mu_{u,j,k},[V_{u,j}]_{k,k})u_{j,k}^2du_{j,k} &+ const \\
&= -J \sum_{k=1}^K \log \rho_{u,k} -\frac{1}{2}\sum_{k=1}^K\rho_{u,k}^{-2}\sum_{j=1}^J \left( [V_{u,j}]_{k,k} + \mu_{u,j,k}^2 \right)&+ const \\
\end{align*}
\begin{align*}
\intertext{($\rho_{u,k}$での微分)=0を考えることにより、}
  - J\rho_{u,k}^{-1} + \rho_{u,k}^{-3}\sum_{j=1}^J \left( [V_{u,j}]_{k,k} + \mu_{u,j,k}^2 \right) = 0 \\
\Leftrightarrow  \rho_{u,k}^2 = \frac{1}{J}\sum_{j=1}^J \left( [V_{u,j}]_{k,k} + \mu_{u,j,k}^2 \right)
\end{align*}
同様に$\ds \rho_{v,k}^2 = \frac{1}{I}\sum_{i=1}^I \left( [V_{v,i}]_{k,k} + \mu_{v,i,k}^2 \right)$も示される。



$\sigma^2$に着目すると、
\begin{align*}
L[q(u_{1:J},v_{i:j});\Lambda_u,\Lambda_v,\sigma^2] &= \int q(u_{1:J})q(v_{1:I}) \sum_{(j,i)\in O}\log p(r_{j,i}|u_j^\top v_i,\sigma^2) du_{1:J}dv_{1:I} &+ const\\
&= -|O|\log \sigma -\frac{1}{2\sigma^2}\int q(u_{1:J})q(v_{1:I}) \sum_{(j,i)\in O} (r_{j,i}-u_j^\top v_i)^2 du_{1:J}dv_{1:I} &+ const\\
\intertext{($\sigma$での微分)=0を考えることにより、}
\sigma^2 & = \frac{1}{|O|}\int q(u_{1:J})q(v_{1:I}) \sum_{(j,i)\in O} (r_{j,i}-u_j^\top v_i)^2 du_{1:J}dv_{1:I}
\end{align*}



\begin{align*}
& \int q(u_{1:J})q(v_{1:I}) \sum_{(j,i)\in O} (r_{j,i}-u_j^\top v_i)^2 du_{1:J}dv_{1:I} \\
= & \sum_{(j,i)\in O} \left( r_{j,i}^2 -2r_{j,i} \int q(u_{1:J})q(v_{1:I}) u_j^\top v_i du_{1:J}dv_{1:I} + T_{j,i} \right)\\
= & \sum_{(j,i)\in O} \left( r_{j,i}^2 -2r_{j,i} \mu_{u,j}^\top \mu_{v,i} + T_{j,i} \right)
\intertext{ただし、}
T_{j,i}=& \int q(u_j)q(v_i) (u_j^\top v_i)^2 du_jdv_i \\
=& \int q(u_j)q(v_i) \mathrm{Tr}(u_ju_j^\top v_iv_i^\top) du_jdv_i
\intertext{独立性により、}
=& \mathrm{Tr}( E_{u_j}[u_ju_j^\top] E_{v_i}[v_iv_i^\top] ) \\
=& \mathrm{Tr}\left\{ (V_{u,j}+\mu_{u,j}\mu_{u,j}^\top)(V_{v,i}+\mu_{v,i}\mu_{v,i}^\top) \right\}
\end{align*}

\end{document}


